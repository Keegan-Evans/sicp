% Options for packages loaded elsewhere
\PassOptionsToPackage{unicode}{hyperref}
\PassOptionsToPackage{hyphens}{url}
%
\documentclass[
]{article}
\usepackage{lmodern}
\usepackage{amsmath}
\usepackage{ifxetex,ifluatex}
\ifnum 0\ifxetex 1\fi\ifluatex 1\fi=0 % if pdftex
  \usepackage[T1]{fontenc}
  \usepackage[utf8]{inputenc}
  \usepackage{textcomp} % provide euro and other symbols
  \usepackage{amssymb}
\else % if luatex or xetex
  \usepackage{unicode-math}
  \defaultfontfeatures{Scale=MatchLowercase}
  \defaultfontfeatures[\rmfamily]{Ligatures=TeX,Scale=1}
\fi
% Use upquote if available, for straight quotes in verbatim environments
\IfFileExists{upquote.sty}{\usepackage{upquote}}{}
\IfFileExists{microtype.sty}{% use microtype if available
  \usepackage[]{microtype}
  \UseMicrotypeSet[protrusion]{basicmath} % disable protrusion for tt fonts
}{}
\makeatletter
\@ifundefined{KOMAClassName}{% if non-KOMA class
  \IfFileExists{parskip.sty}{%
    \usepackage{parskip}
  }{% else
    \setlength{\parindent}{0pt}
    \setlength{\parskip}{6pt plus 2pt minus 1pt}}
}{% if KOMA class
  \KOMAoptions{parskip=half}}
\makeatother
\usepackage{xcolor}
\IfFileExists{xurl.sty}{\usepackage{xurl}}{} % add URL line breaks if available
\IfFileExists{bookmark.sty}{\usepackage{bookmark}}{\usepackage{hyperref}}
\hypersetup{
  hidelinks,
  pdfcreator={LaTeX via pandoc}}
\urlstyle{same} % disable monospaced font for URLs
\setlength{\emergencystretch}{3em} % prevent overfull lines
\providecommand{\tightlist}{%
  \setlength{\itemsep}{0pt}\setlength{\parskip}{0pt}}
\setcounter{secnumdepth}{-\maxdimen} % remove section numbering
\ifluatex
  \usepackage{selnolig}  % disable illegal ligatures
\fi

\author{}
\date{}

\begin{document}

Given a set of coin denominations \(\mathbb{C}\) of size \(n\), in how
many ways can an amount \(A\) be changed using the coin denominations in
\(\mathbb{C}\)?

A fairly straightforward solution to this is as follows, using the set
of 5 coin denominations, \$ \left( 1,5,10,25,50 \right) \$.

\begin{verbatim}
(define (count-change amount)
  (cc amount 5))

(define (nth xs n)
  (first (drop xs n)))

(define (denom n)
  (nth '(1 5 10 25 50) (- n 1)))

(define (cc amount kinds-of-coins)
  (cond ((= amount 0) 1)
        ((or (= kinds-of-coins 0)
             (< amount 0)) 0)
        (else (+ (cc amount (- kinds-of-coins 1))
                 (cc (- amount
                        (denom kinds-of-coins))
                     kinds-of-coins)))))
\end{verbatim}

Drawing the call-tree of (count-change 11) is straightforward using the
substitution method. The later part of Exercise 1.14 of SICP asks you to
find the orders of growth for the space and time consumed by the
procedure cc.

Space complexity The space consumed by the recursive process generated
by cc is going to be proportional to the maximum height of the recursion
tree corresponding to an instance of cc, since at any given point in the
recursive process, we must only keep track of the trail of nodes that
leads to the root of the tree.

The maximum height of the call tree, for relatively larger amounts
\(n\), is going to be dominated by the subtree that contains successive
recursive calls with the amount decreased by 1. Clearly, this means the
maximum height is going to be linear in the amount \(n\), or
\(\Theta \left(n\right)\).

Time complexity Let us start with the call tree for changing some amount
\(n\), with just 1 kind of coin, i.e., the call tree for (cc n 1):

Call tree for (n, 1)

We are only allowed here to use one kind of coin, with value
\(\mathbb{C}_{1} = 1\).

The red nodes are terminal nodes that yield \(0\), the green node is a
terminal node that yields \(1\) (corresponding to the first condition in
the code for cc). Each nonterminal node spawns two calls to cc, one (on
the left) with the same amount, but fewer kinds of coins, and the other
(on the right) with the amount reduced by 1 and equal kinds of coins.

Excluding the root, each level has exactly \(2\) nodes, and there are
\(n\) such levels. This means, the number of cc calls generated by a
single (cc n 1) call (including the original call) is:

\[ T\left(n,1\right) = 2n + 1 = \Theta \left(n\right) \] Next, we will
look at the call tree of (cc n 2) to calculate \(T\left(n,2\right)\):

Call tree for (n, 1)

Here, we are allowed to use two denominations of coins,
viz.~\(\mathbb{C}_{2} = 5\) and \(\mathbb{C}_{1} = 1\).

Each black node spawns a (cc m 1) subtree (blue), which we've already
analyzed, and a (cc (- m 5) 2) subtree. The node colored in red and
green is a terminal node, but yields \(0\) if the amount is less than
zero and \(1\) if the amount is exactly zero. I've denoted this final
amount as \(\epsilon\), which can be \(\le0\).

Excluding the root and and the last level in this tree which contains
the red-green terminal node, there will be exactly
\(\lfloor {\frac {n} {5} } \rfloor\) levels. Now each of these levels
contains a call to (cc m 1) (the blue nodes), each of which, in turn, is
\(\Theta\left(n\right)\) in time. So each of these levels contains
\(T\left(n,1\right) + 1\) calls to cc. Therefore, the total number of
nodes (including the terminal node and the root) in the call tree for
(cc n 2) is:

\[ T\left(n,2\right) = \lfloor {\frac {n} {5} } \rfloor \left( T\left(n,1\right) + 1\right) + 2 = \lfloor {\frac {n} {5} } \rfloor \left( 2n + 2 \right) + 2 = \Theta\left(n^2\right) \]
Moving ahead, let's take a look at the call tree of (cc n 3), i.e., we
are now allowed to use three denominations of coins, the new addition
being \(\mathbb{C}_{3} = 10\):

Call tree for (n, 1)

Here also, we see, similar to the previous case, that the total number
of calls to cc will be

\[ T\left(n,3\right) = \lfloor {\frac {n} {10} } \rfloor \left( T\left(n,2\right) + 1 \right) + 2 = \lfloor {\frac {n} {10} } \rfloor \times \Theta\left(n^2\right) + 2 = \Theta\left(n^3\right) \]
We can see a pattern here. For some \(k\), \(k \gt 1\), we have,

\[ T\left(n,k\right) = \lfloor {\frac {n} { \mathbb{C}_{k} } } \rfloor \left( T\left(n, k-1\right) + 1 \right) + 2 \]
Here, \(\mathbb{C}_{k}\) is the \(k^{th}\) coin denomination. We can
expand this further:

\[ T\left(n,k\right) = \lfloor {\frac {n} { \mathbb{C}_{k} } } \rfloor \left( T\left(n, k-1\right) + 1 \right) + 2 = \lfloor {\frac {n} { \mathbb{C}_{k} } } \rfloor \left( \lfloor {\frac {n} { \mathbb{C}_{k-1} } } \rfloor \left(... \lfloor \frac {n} { \mathbb{C}_{2} } \rfloor \left(2n+1\right) ...\right) \right) + 2 = \Theta\left(n^k\right) \]
Conclusion In the preceding analysis of the recursive process generated
by cc, we see that although it is an elegant and straighforward way of
solving the problem, it is not particularly efficient in time and grows
exponentially with the number of allowed denominations of coins, and
polynomially with the amount to be changed when the number of
denominations is fixed. Note that the actual values of the coin
denominations have no effect on the order of growth of this process.

\end{document}
